% Following magic comments allow for compilation of root file
% !TEX root = ../../../temp_manuscript.tex
\chapter{Acknowledgements}
\begin{ChapterAbstractNoTitle}
\end{ChapterAbstractNoTitle}

I once read that the acknowledgements sections is the only piece of writing in which a scientist can be person with feelings, rather than just an objective observer.
Hence, it seems weird that the acknowledgements section is just such a small part of a thesis, as my PhD trajectory was definitely a journey of feelings.
To be completely honest, while I'm writing this and finishing up the last few things for my thesis I'm still not entirely convinced that this is really the end, I guess that is going to take a few more months.
Although it is almost a cliche by now to say that doing a PhD is an intense experience, both in a personal and in a work experience sense, it is hard to describe what exactly it is that makes it feel like this.
There are of course the factors like deadlines, the inert competition present in academia, the sometimes solitary nature of the work (even when working in a group), and the pressure to put your PhD first.
However, there seems to be a more fundamental, underlying experience that is in some strange way both immensely personal but also shared by everybody doing a PhD.
Whatever happens, do not forget to love yourself, love the people around you and make sure you are surrounded by people that love you.

Of course, doing a PhD is not something that you can do on your own, and there are numerous people to thank.
I would like to start with my supervisors.

\textbf{Marion Smits}, thank you for your helpful clinical insights and keeping sight on the clinical applicability of the research.
You always knew how to add the perfect finishing touches to everything, whether it was a paper, figure, or presentation, and I was also amazed by how much you could improve the quality of something with just a few minor changes.
I would also like to thank you for your help on a personal level, it definitely helped me through some rougher times, and your openness and honesty is something I very much appreciated.
When the inevitable science revolution will finally happen, I hope you will be at the forefront leading the troops.

\textbf{Stefan Klein}, thank you for the intense discussions, and the nitpicking every detail of an analysis or idea, you could see the smallest details even when I explained something only at a very high level.
Whether intentional or unintentional, you always said something along the lines of \say{I don't think it is possible to do it like this} whenever I needed to hear it, as then proving that it could actually be done was a very good motivator for me.
Although it could be frustrating at times when something indeed did end up not working, I think the challenge you provided did bring a lot of the work to greater heights.

\textbf{Wiro Niessen}, thank you for keeping the overarching goal of completing my thesis in mind, and steering where necessary.
Thank you for welcoming me to BIGR, and giving me the opportunity to do my PhD in such an exciting group.

At least as important are of course my direct colleagues, without whom this work would never have happened.

First of all, \textbf{Fatih}, thank you for being my (clinical) partner in crime during this project.
Without your help I would have been completely lost in the world of neuro-oncology lingo, and would have missed all the important subtleties.
Your drive \dots


My roommates over the years: \textbf{Zahra}, \textbf{Gerda}, \textbf{Arno}, \textbf{Florian}, and \textbf{Gijs}.
It was always a lot of fun to share a room with you, chatting about everything and nothing \dots.

My fellow committee members over the years.
From YOUNG Medical Delta: \textbf{Demi}, \textbf{Eline}, \textbf{Emilie}, \textbf{Flip}, \textbf{Isabelle}, \textbf{Julia}, \textbf{Kick}, \textbf{Otto}, \textbf{Thijs}, \textbf{Teun}, and \textbf{Simone}.
It was a lot of fun to work together with you and set up new initiatives, tackling challenges outside of my daily work and seeing more in the field of medical technology outside of my own.
\textbf{Edoardo}, \textbf{Isabelle}, \textbf{Lana}, \textbf{Lisanne}, \textbf{Merel}, organizing the Dutch Hacking Health Rotterdam with you was a very nice experience.
There was a lot of work, and it was an intense, but very exciting weekend.
From Promeras: \textbf{Daphne}, \textbf{Elisabeth}, \textbf{Merel}, \textbf{Pauline}, \textbf{Rama}, \textbf{Sarah}, \textbf{Steffi}, \textbf{Robbin}, \textbf{Xavier}.
It was very interesting to explore the inner works of the Erasmus MC, and to get into contact with PhD students from all kinds of departments that I would otherwise never have met.


My friends from The Hague: \textbf{Georgina}, \textbf{Jason}, and \textbf{Onur}.
Thank you for listening to my rants whenever I had a bad day at work, and for the board game nights and digital game nights to distract me.


\textbf{Mam en pap}, zonder jullie stond ik hier nu niet.
Bedankt voor alle liefde die jullie mij van kinds af aan al hebben gegeven, en voor jullie onvoorwaardelijke geloof in mij.
Jullie zijn de basis vanuit waar ik heb kunnen groeien, en er zullen nooit genoeg woorden zijn om te zeggen hoe dankbaar ik jullie daarvoor ben.

Mam, bedankt voor jouw ongelofelijk goede voorbeeld van liefde en warmte voor mensen, ik probeer dit elke dag te volgen en door te geven.
Jouw (soms ietwat flauwe) humor was vaak precies wat ik nodig had na een week waarin alles net niet lukte.

Pap, jouw analytische skill en, goede gewoonte om alles nog eens dubbel te checken hebben me vaak geholpen tijdens mijn werk.
Hierdoor heb ik vaak fouten kunnen voorkomen waar ik anders overheen had gekeken.

Lastly \textbf{Pauline}, now matter how hard this PhD was I would do it all again if it means I could meet you and see you walk into my office for the first time again.
You have shown me that it is important and okay to love myself, and managed to get out a side of me that I though was lost.
You definitely got me through some of the rougher times, but most importantly we shared most of the happiest times.
Thank you for seeing me.
