% Following magic comments allow for compilation of root file
% !TEX root = ../../../temp_manuscript.tex
\chapter{Acknowledgements}
\begin{ChapterAbstractNoTitle}
\end{ChapterAbstractNoTitle}

I once read that the acknowledgements section is the only piece of writing in which a scientist can be a person with feelings, rather than just an objective observer.
Hence, it seems weird that the acknowledgements section is such a small part of a thesis since my PhD trajectory was definitely an emotional journey.
To be completely honest, while I'm writing this and finishing up the last few things for my thesis, I'm still not entirely convinced that this is really the end.
I guess that is going to take a few more months.
Although it is almost a clich{\'e} by now to say that doing a PhD is an intense experience, both in a personal sense and in terms of work experience, it is hard to describe precisely what it is that makes it feel like this.
There are, of course, factors like deadlines, the inherent competition present in academia, the sometimes solitary nature of the work (even when working in a group), and the pressure to put your PhD first.
However, there seems to be a more fundamental, underlying experience that is in some strange way both immensely personal but also shared by everybody doing a PhD.
And I think it is this experience, more than any obtained knowledge, that I will take with me in future endeavours.

Of course, doing a PhD is not something that you do all by yourself, and there are numerous people to thank.
I would like to start with my supervisors.

\textbf{Marion}, thank you for your helpful clinical insights and for keeping sight of the clinical applicability of the research.
You always knew how to add the perfect finishing touches to everything, whether it was a paper, figure, or presentation.
I was time and again amazed by how much you could improve the quality of something with just a few minor changes.
I would also like to thank you for your support on a personal level, it definitely helped me through some rougher times.
Your openness and honesty are something I appreciated very much.
When the inevitable science revolution finally happens, I hope you will be at the forefront leading the troops.

\textbf{Stefan}, thank you for the intense discussions and for nitpicking every detail of an analysis or idea that I presented.
You could see the smallest details even when I explained something only at a very high level.
Whether intentional or unintentional, you always said something along the lines of \say{I don't think it is possible to do it like this} whenever I needed to hear it.
Proving that it could actually be done was the perfect motivator for me to keep trying.
Although it could be frustrating when something indeed did end up not working, I think the challenge you provided brought a lot of the work to greater heights.

\textbf{Wiro}, thank you for keeping the overarching goal of completing my thesis in mind and steering where necessary.
Thank you for welcoming me to BIGR and giving me the opportunity to work on my PhD in such a stimulating group.

Of course, at least as important are my direct colleagues, without whom this work would never have happened.

First of all, \textbf{Fatih}, thank you for being my (clinical) partner in crime during this project.
Without your help, I would have been completely lost in the world of neuro-oncological terminology and I would have missed a lot of important subtleties.
Your drive to keep pushing for the best possible research that we could do and tireless efforts to keep collecting data and segmenting tumours was a major contribution to the level of research that we presented and that I'm very proud of.

\textbf{Maarten}, from a \say{simple and quick} experiment we went to quite an extensive paper in the end.
It was very informative to see your perspective from the genetics side and I was always very impressed by your skills to create nice, clear figures.

\textbf{Martijn}, starting up the \say{radiomics} section within BIGR with you was quite an adventure.
In the beginning, it was a relatively unknown term, but now it has grown out to be quite a concept, and I'd like to think we helped that along a bit.

\textbf{Karin}, it was nice to have a bit of extra technical counterweight to all of the clinicians in the neuro-onco group.
I appreciated your more mathematical view on things and your honest and direct opinion on the newest, flashiest methods.

\textbf{George} and \textbf{Renske}: thank you for your efforts in collecting data and segmenting many, many scans.

\textbf{Arnaud}, \textbf{Erik}, \textbf{Geert}, \textbf{Joost}, \textbf{Martin}, \textbf{Philip}, \textbf{Pim}, \textbf{Rishi}, and \textbf{Roelant}: thank you for your help in all of the different projects.
Without your contributions, specific expertise, and feedback, it would have been impossible to take up this challenge of interdisciplinary research.
A special thanks to Rishi, Geert, Philip, and Roelant for their willingness to share the data from their institutes.

My students over the years: \textbf{Dai}, \textbf{Mar{\'\i}a}, \textbf{Mink}, and \textbf{Paul}.
Thanks to your contributions, I was able to explore more ideas and possibilities for research.
Supervising you was a real pleasure and a good learning experience for me as well.

The people from BIGR\@:
\textbf{Jose} you grew from one of my students to a colleague and a friend.
I'm glad you decided to accept the project, as this lead to the fun gaming moments later one.
\textbf{Zahra}, your kindness is very infectious.
No matter what happened, you would constantly have a smile on your face and often have some lovely food to share.
\textbf{Florian}, I was never sure whether you were strongly convinced by what you were saying or whether you just wanted to win the argument, but you were always intense in what you said and did.
Although the short time we shared an office might not have been very productive, it was a lot of fun.
\textbf{Gijs}, you often managed to surprise me with everything you were knowledgeable about, and you always had something interesting to add to any conversation.
Your sense of humor might be somewhat hidden, but when it came out, I appreciated it very much.
\textbf{Arno}, in my mind, you're still a bit the student that I shared my first office with, even though you've already long outgrown that position.
For me, it feels like your enthusiasm and boundless energy are incompatible with being a PhD student.
\textbf{Vikram}, you always properly thought things over, which lead to deeper insights, and you were open to help out with whatever someone needed.

Thank you to all of my other colleagues with whom I shared many lunches, coffee moments, and boardgame evenings:
\textbf{Annegreet}, \textbf{Antonio}, \textbf{Dirk}, \textbf{Esther B.}, \textbf{Gennady}, \textbf{Gerda}, \textbf{Gokhan}, \textbf{Henri}, \textbf{Hua}, \textbf{Jifke}, \textbf{Kasper}, \textbf{Mahlet}, \textbf{Riwaj}, \textbf{Shuai}, \textbf{Theo}, \textbf{Thomas}, \textbf{Willem}, \textbf{Yao}, \textbf{Yuan Yuan}, and many more over the years.
To all the people who have attended the board game evenings: I had a lot of fun being able to share my hobby with you and even, in some cases, introduce other people to that hobby.

Also a big thanks to the MR physics department, \textbf{Esther W.}, \textbf{Fatemeh}, and \textbf{Juan}, to name but a few.
Thanks for answering my (probably very basic) MR-related questions, especially when I came in as a newbie and couldn't wrap my head around all of the concepts.

A big thanks to the members of the infrastructure group: \textbf{Adriaan}, \textbf{Hakim}, and \textbf{Marcel}.
Your help with XNAT, the cluster, and other infrastructure was essential for all the work in this thesis.

\textbf{Jeffrey}, \textbf{Laurens}, and \textbf{Mart}, thanks for all the help with the IT and data collection for the different studies.

\textbf{Annemarijn}, \textbf{Desiree}, and \textbf{Petra}, thanks for all the help with the administrative side of things over the years.

\textbf{Sebastiaan}, \textbf{Sebestien}, and \textbf{Sabastian}, although we never met, I received a lot of your emails, and it seems like you did quite some work for me.
Please let me know if you ever want to have those emails back.

My fellow committee members over the years.
From YOUNG Medical Delta: \textbf{Demi}, \textbf{Eline}, \textbf{Emilie}, \textbf{Flip}, \textbf{Isabelle}, \textbf{Julia}, \textbf{Kick}, \textbf{Otto}, \textbf{Thijs}, \textbf{Teun}, and \textbf{Simone}.
It was a lot of fun to work together with you, setting up new initiatives, tackling challenges outside of my daily work, and discovering more of the field of medical technology outside of my own work.
\textbf{Edoardo}, \textbf{Isabelle}, \textbf{Lana}, \textbf{Lisanne}, \textbf{Merel}, organizing the Dutch Hacking Health Rotterdam with you was a very nice experience.
There was a lot of work, and it was an intense but very exciting weekend.
From Promeras: \textbf{Daphne}, \textbf{Elisabeth}, \textbf{Merel}, \textbf{Pauline}, \textbf{Rama}, \textbf{Sarah}, \textbf{Steffi}, \textbf{Robbin}, \textbf{Xavier}.
It was very interesting to explore the inner workings of the Erasmus MC and to get in contact with PhD students from all kinds of departments that I would otherwise never have met.

My friends from The Hague: \textbf{Georgina}, \textbf{Jason}, and \textbf{Onur}.
Thank you for listening to my rants whenever I had a bad day at work (and sometimes even enjoying them) and for the very enjoyable board game nights and digital game nights.
Thank you for your feedback on my introduction, making it a better fit for a general audience.
Georgina and Onur, sharing the PhD journey over the years was a helpful experience, both for the good and bad times.
Jason, thanks for sharing my humor (so at least I'm not alone) and making me laugh till I can't breathe.
Georgina, thanks for reading these acknowledgements (and many other documents over the years) as a final check.

The Delft group: \textbf{In{\'e}s}, \textbf{Lana}, and \textbf{Liesbeth}.
Thanks for the nice get-togethers and cooking evenings.
Lana en Ines, thank you for the coffee moments at Erasmus and our discussions about doing a PhD. I'm missing the PhD buddies already.

Of course my paranymphs: \textbf{Joep} and \textbf{Kim}.
Thanks for helping me with all the organizational things surrounding my defense, on top of all the fun moments and being all-around awesome people.

\textbf{Opa en Oma}, van de kinderkoffie en gesneden boterhammen tot een PhD, wie had dat verwacht.
Bedankt voor al jullie liefde en goede zorgen over de jaren!

\textbf{Mam en pap}, zonder jullie stond ik hier nu niet.
Bedankt voor alle liefde die jullie mij van kinds af aan al hebben gegeven en voor jullie onvoorwaardelijke geloof in mij.
Jullie zijn de basis vanuit waar ik heb kunnen groeien, er zullen nooit genoeg woorden zijn om te zeggen hoe dankbaar ik jullie daarvoor ben.

Mam, bedankt voor jouw ongelofelijk goede voorbeeld van liefde en warmte voor mensen, ik probeer dit voorbeeld elke dag te volgen en door te geven.
Jouw (soms ietwat flauwe) humor was vaak precies wat ik nodig had na een week waarin alles net niet lukte.

Pap, jouw analytische vaardigheden en goede gewoonte om alles nog eens dubbel te checken hebben me vaak geholpen tijdens mijn werk.
Hierdoor heb ik vaak fouten kunnen voorkomen die ik anders over het hoofd had gezien.

Lastly \textbf{Pauline}, no matter how hard this PhD was, I would do it all again if it would give me another chance to meet you and see you walk into my office for the first time.
You have shown me that it is important and okay to love myself and managed to get out a side of me that I thought was lost.
You definitely got me through some of the rougher times, but most importantly, we shared some of the happiest times.
Thank you for seeing me.

\vspace*{10em}


\hfill \textit{\epigraph{\say{No, I can't do this all on my own. I know that I'm no Superman}}{--- \textup{Lazlo Bane}, intro song to Scrubs}} \hfill
