\chapter[Radiomics: Data mining using quantitative medical image features][Radiomics]{Radiomics: Data mining using quantitative medical image features}

\begin{ChapterAbstract}
Radiomics uses multiple image features from medical imaging data to predict clinical variables.
Various features can be constructed to describe the properties of the full image, or those of a specific region of interest such as a tumor.
These features may be related to a wide variety of clinical variables, such as disease characteristics, genetics and therapy response.
This can be done through the use of machine learning, which enables the training of a model on these features using data of patients for which the relevant clinical variables are already known.
The resulting models may be used as a diagnostic aid for the prediction of labels such as tumor phenotype and therapy response in new patients.
Thereby, radiomics can provide a non-invasive alternative for invasive procedures, such as biopsies, to uncover disease characteristics or clinical outcomes.
Radiomics therefore has a high potential to be a valuable tool for clinical practice.
This may explain the rise in popularity of radiomics in the medical imaging research field in recent years, resulting in many methods and applications.
This chapter provides a guide through the several aspects of designing a radiomics study.

\publishedas{starmans2020radiomics}
\end{ChapterAbstract}

Due to copyright restrictions the full text of this chapter cannot be shared publicly. It is available at \url{https://www.sciencedirect.com/science/article/pii/B9780128161760000235}
