%%%%%%%%%%%%%%%%%%%%%%%
% Defined environments
%%%%%%%%%%%%%%%%%%%%%%%

\newenvironment{ChapterAbstract}
{
% We reset all used abbreviations
\glsresetall
\vspace*{\fill}
\begin{adjustwidth}{2cm}{2cm}
% Add 'Abstract' as title
\begin{center}
\textbf{Abstract}
\end{center}
\vspace{2em}
% To italics and footnotesize
\itshape\small
}
{
\end{adjustwidth}
\vspace{\fill}
% Afterwards we once again reset all abbreviations before using them in the text
\glsresetall
\clearpage
}

\newenvironment{ChapterAbstractNoTitle}
{
% We reset all used abbreviations
\glsresetall
\begin{adjustwidth}{2cm}{2cm}
% Add 'Abstract' as title
% To italics and footnotesize
\itshape\small
}
{
\end{adjustwidth}
% Afterwards we once again reset all abbreviations before using them in the text
\glsresetall
}
%%%%%%%%%%%%%%%%%%%%%%%
% Chapter page layout
%%%%%%%%%%%%%%%%%%%%%%%

\makeatletter
\titleformat{\chapter}[display]
{
\if@mainmatter
    % This is for chapter layout in mainmatter
    \vfill
\fi
}
{
\if@mainmatter
    % This is for chapter layout in mainmatter
    % \LARGE\MakeUppercase{\chaptertitlename}~\thechapter
\fi
}
{
\if@mainmatter
    % This is for chapter layout in mainmatter
    5pt
\else
    % This is for chapter layout in front and back matter
    1pt
\fi
}
{
\if@mainmatter
    % This is for chapter layout in mainmatter
    \Huge #1
    % \thispagestyle{epigraph}
\else
    % This is for chapter layout in front and back matter
    \Huge\bfseries #1
    % \thispagestyle{empty}
\fi
}
[
\if@mainmatter
    % This is for chapter layout in mainmatter
    \vfill
    % \newpage
\fi
]
\makeatother

% Settings for epigraphs (quotes at beginning of chapters)
\epigraphfontsize{\small\itshape}
\setlength\epigraphwidth{8cm}
\setlength\epigraphrule{0pt}


%%%%%%%%%%%%%%%%%%%%%%%
% Section layouts
%%%%%%%%%%%%%%%%%%%%%%%
% To also number subsections
\setsecnumdepth{subsection}
\settocdepth{section}

% Sub appendices get a specific figure and table count
\AtBeginEnvironment{subappendices}{%
\section*{Appendix}
\counterwithin{figure}{section}
\counterwithin{table}{section}
%Also change the names from sections and subsections
\crefalias{section}{appendix}
\crefalias{subsection}{appendix}
}

\AtEndEnvironment{subappendices}{%
\counterwithout{figure}{section}
\counterwithout{table}{section}
\crefalias{section}{section}
\crefalias{subsection}{subsection}
}


%%%%%%%%%%%%%%%%%%%%%%%
% Bibliography
%%%%%%%%%%%%%%%%%%%%%%%

% % Ensure that bibliography not in table of contents unless specifically requested
\nobibintoc

% % Bold your own name in the bibliography of publications
\AtEveryBibitem{
\renewcommand*{\mkbibnamegiven}[1]{%
  \ifitemannotation{highlight}{\textbf{#1}}{#1}}
}

\AtEveryBibitem{
\renewcommand*{\mkbibnamefamily}[1]{%
  \ifitemannotation{highlight}
    {\ifpartannotation{authorship}{shared}{\textbf{#1}\textsuperscript{*}}
    {\ifitemannotation{presenter}{\textbf{#1}\textsuperscript{$\dagger$}}{\textbf{#1}}}}
  {\ifpartannotation{authorship}{shared}{#1\textsuperscript{*}}
  {\ifitemannotation{presenter}{#1\textsuperscript{$\dagger$}}{#1}}}
}}

\AtEveryBibitem{
\renewcommand*{\mkbibnameprefix}[1]{%
  \ifitemannotation{highlight}
    {\textbf{#1}}
    {#1}}
}

\AtEveryBibitem{
\renewcommand*{\mkbibnamesuffix}[1]{%
  \ifitemannotation{highlight}
    {\textbf{#1}}
    {#1}}
}

\renewcommand*{\mkbibnamegiven}[1]{%
  \ifitemannotation{highlight}
    {\textbf{#1}}
    {#1}}

\renewcommand*{\mkbibnamefamily}[1]{%
  \ifitemannotation{highlight}
    {\ifpartannotation{authorship}{shared}{\textbf{#1}\textsuperscript{*}}{\textbf{#1}}}
  {\ifpartannotation{authorship}{shared}{#1\textsuperscript{*}}{#1}}}

\renewcommand*{\mkbibnameprefix}[1]{%
  \ifitemannotation{highlight}{\textbf{#1}}{#1}}

\renewcommand*{\mkbibnamesuffix}[1]{%
  \ifitemannotation{highlight}{\textbf{#1}}{#1}}

% We remove URL from the bib, unless it is misc
\DeclareStyleSourcemap{
  \maps[datatype=bibtex, overwrite=true]{
    \map{
      \step[fieldsource=url, final]
      \step[typesource=misc, typetarget=online]
    }
  }
}

% We do not want the day in the date of a publication
\AtEveryBibitem{\clearfield{day}}

% We also do not want the arxiv category to be printed
\AtEveryBibitem{\clearfield{eprintclass}}

% For some journal we don't have pages but article numbers
% Fix that here

\NewBibliographyString{artno}
\DefineBibliographyStrings{english}{artno = {art\adddotspace no\adddot}}

\DeclareFieldFormat[article,periodical]{eid}{\bibstring{artno}\addabbrvspace #1}

\NewBibliographyString{abstractno}
\DefineBibliographyStrings{english}{abstractno = {abstract no\adddot}}

\DeclareFieldFormat[inproceedings]{eid}{\bibstring{abstractno}\addabbrvspace #1}

% In proceedings, the series we use for things like abstract books, but not nice to print'ser' before
% This removes that
\DeclareFieldFormat[inproceedings]{series}{#1}

% Some page numbers have letters in them, we need to add them to numchars
% in order to get biblatex to print 'p.' or 'pp.' before
\DeclareNumChars{SeE}

% We only want the ISBN if there is no DOI
\DeclareSourcemap{
  \maps[datatype=bibtex]{
     \map{
        \step[fieldsource=doi,final]
        \step[fieldset=isbn,null]
        \step[fieldset=issn,null]
        }
      }
}



% This is to specify abstract numbers in cases they don't have a direct reference
\renewbibmacro*{chapter+pages}{%
  \printfield{chapter}%
  \setunit{\bibpagespunct}%
  \printfield{pages}%
  \setunit{\bibpagespunct}%
  \printfield{eid}
  \newunit}
% For PMID articles

% \DeclareDatamodelFields[type=field,datatype=verbatim]{pmid}
% \DeclareDatamodelEntryfields{pmid}
% \DeclareDatamodelFields[type=field,datatype=verbatim]{pmcid}
% \DeclareDatamodelEntryfields{pmcid}
\DeclareFieldFormat{eprint:pmcid}{%
  PMCID\addcolon\space
  \ifhyperref
    {\href{https://pubmed.ncbi.nlm.nih.gov/#1}{\nolinkurl{#1}}}
    {\nolinkurl{#1}}}
\DeclareFieldAlias{eprint:PMC}{eprint:pmcid}
\DeclareFieldAlias{eprint:PMCID}{eprint:pmcid}
\DeclareFieldAlias{eprint:pmc}{eprint:pmcid}
\DeclareFieldAlias{pmcid}{eprint:pmcid}
\DeclareFieldAlias{pmid}{eprint:pubmed}

\DeclareFieldFormat{eprint:acm}{%
  ACM\space DOI\addcolon\space
  \ifhyperref
    {\href{https://dl.acm.org/doi/#1}{\nolinkurl{#1}}}
    {\nolinkurl{#1}}}



% In unpublished also print arxiv
\DeclareBibliographyDriver{unpublished}{%
  \usebibmacro{bibindex}%
  \usebibmacro{begentry}%
  \usebibmacro{author}%
  \setunit{\labelnamepunct}\newblock
  \usebibmacro{title}%
  \newunit
  \printlist{language}%
  \newunit\newblock
  \usebibmacro{byauthor}%
  \newunit\newblock
  \printfield{howpublished}%
  \newunit\newblock
  \printfield{note}%
  \newunit\newblock
  \usebibmacro{location+date}%
%  \newunit\newblock% DELETED
%  \iftoggle{bbx:url}% DELETED
%    {\usebibmacro{url+urldate}}% DELETED
%    {}% DELETED
  \newunit\newblock% NEW
  \usebibmacro{doi+eprint+url}% NEW
  \newunit\newblock
  \usebibmacro{addendum+pubstate}%
  \setunit{\bibpagerefpunct}\newblock
  \usebibmacro{pageref}%
  \usebibmacro{finentry}}






%%%%%%%%%%%%%%%%%%%%%%%
% OTHER
%%%%%%%%%%%%%%%%%%%%%%%

% Make sure that all are numbers are formatted the same
\sisetup{
  exponent-product=\ensuremath{\cdot},
  group-separator={,},
  output-decimal-marker = {.},
  range-phrase = --,
  output-product=\cdot,
  detect-all = true
}
\DeclareSIUnit{\nof}{\#~of}

\makeatletter
\DeclareCaptionTextFormat{periodcheck}{#1\@addpunct{.}}
\makeatother


% We set up the captions
\captionsetup{%
format=hang,
textformat=periodcheck,
font={it},
labelfont={bf},
width=0.9\textwidth
}

% Remove brackets around equation numbers
\makeatletter
\renewcommand\tagform@[1]{\maketag@@@{\ignorespaces#1\unskip\@@italiccorr}}
\makeatother
\creflabelformat{equation}{#2\textup{#1}#3}

% For oxford comma in cleverref
\newcommand{\creflastconjunction}{, and\nobreakspace}

% Ensure that we don't have weird spacing between paragraphs
\raggedbottom



%%%%%%%%%%%%%%%%%%%%%%%%%%%%%%%%%%%%%%%%%
% Thumb index
%%%%%%%%%%%%%%%%%%%%%%%%%%%%%%%%%%%%%%%%%

\def\cleardoublepage{%
    \clearpage%
    \if@twoside%
        \ifodd\c@page%
        \else%
            \thispagestyle{empty}%
            \vspace*{\fill}%
            \newpage%
        \fi%
    \fi%
}

\if@print%
    \newcommand*\cropmarks{%
        \ifodd\c@page%
            \begin{tikzpicture}[remember picture,overlay]
                \draw ($(current page.north east)+(0mm,-3mm)$) -- ($(current page.north east)+(-2mm,-3mm)$);
                \draw ($(current page.north east)+(-3mm,0mm)$) -- ($(current page.north east)+(-3mm,-2mm)$);
                \draw ($(current page.south east)+(0mm,3mm)$) -- ($(current page.south east)+(-2mm,3mm)$);
                \draw ($(current page.south east)+(-3mm,0mm)$) -- ($(current page.south east)+(-3mm,2mm)$);
            \end{tikzpicture}%
        \else%
            \begin{tikzpicture}[remember picture,overlay]
                \draw ($(current page.north west)+(0mm,-3mm)$) -- ($(current page.north west)+(2mm,-3mm)$);
                \draw ($(current page.north west)+(3mm,0mm)$) -- ($(current page.north west)+(3mm,-2mm)$);
                \draw ($(current page.south west)+(0mm,3mm)$) -- ($(current page.south west)+(2mm,3mm)$);
                \draw ($(current page.south west)+(3mm,0mm)$) -- ($(current page.south west)+(3mm,2mm)$);
            \end{tikzpicture}%
        \fi%
    }
\else
    \newcommand*\cropmarks{}
\fi%


%%% Thumb indices consist of white text on a rectangular colored background. The
%%% font-size is 75% of the size of thumb height.
\newif\ifthumb
\colorlet{thumb}{gray}
\newlength\thumbheight
\setlength\thumbheight{24pt}
\newlength\thumbedge
\setlength\thumbedge{4pt}
\newlength\thumbhspace
\setlength\thumbhspace{36pt}
\newlength\thumbvspace
\setlength\thumbvspace{2\thumbheight}

\newlength\thumbwidth
\setlength\thumbwidth{36pt}
\newlength\thumbspacing
\setlength\thumbspacing{2\thumbheight}


%% We need the TikZ library calc to calculate the coordinates of the thumb
%% indices.
\usetikzlibrary{calc}


%% The lthumb command prints the current chapter number in a thumb index on the
%% left (even) page.
\newcommand*\lthumb{%
    \ifthumb%
        \begin{tikzpicture}[remember picture,overlay]
            \coordinate (top margin) at (0pt,1in+\topmargin+\headheight+\headsep);
            \coordinate (left margin) at (1in+\evensidemargin+4mm,0pt);
            %% Calculate the corners of the thumb index based on the current
            %% chapter number.
            \coordinate (top left) at ($(current page.north west)-(top margin)-(0pt,\value{chapter}\thumbvspace-\thumbvspace)$);
            \coordinate (bottom right) at ($(top left)+(left margin)-(\thumbhspace,\thumbheight)$);
            %% Shift the left edge to prevent the rounded corner from showing.
            \coordinate (top left) at ($(top left)-(\thumbedge,0pt)$);
            %% Draw the thumb index.
            \fill[fill=thumb,rounded corners=\thumbedge](top left) rectangle (bottom right);
            %% Print the chapter number at the center right in the thumb index.
            \coordinate (center right) at ($(bottom right)+(0pt,0.5\thumbheight)$);
            \node at (center right)[anchor=east,inner sep=2\thumbedge]{
                \bfseries\color{white}
                \fontsize{0.75\thumbheight}{0.75\thumbheight}\selectfont
                \thechapter
            };
        \end{tikzpicture}%
    \fi%
}

%% rthumb draws a thumb index on the right (odd) page.
\newcommand*\rthumb{%
    \ifthumb%
        \begin{tikzpicture}[remember picture,overlay]
            \coordinate (top margin) at (0pt,1in+\topmargin+\headheight+\headsep);
            \coordinate (right margin) at (1in+\evensidemargin+4mm,0pt);
            %% Calculate the corners of the thumb index based on the current
            %% chapter number.
            \coordinate (top right) at ($(current page.north east)-(top margin)-(0pt,\value{chapter}\thumbvspace-\thumbvspace)$);
            \coordinate (bottom left) at ($(top right)-(right margin)-(-\thumbhspace,\thumbheight)$);
            %% Shift the left right to prevent the rounded corner from showing.
            \coordinate (top right) at ($(top right)+(\thumbedge,0pt)$);
            %% Draw the thumb index.
            \fill[fill=thumb,rounded corners=\thumbedge](top right) rectangle (bottom left);
            %% Print the chapter number at the center right in the thumb index.
            \coordinate (center left) at ($(bottom left)+(0pt,0.5\thumbheight)$);
            \node at (center left)[anchor=west,inner sep=2\thumbedge]{
                \bfseries\color{white}
                \fontsize{0.75\thumbheight}{0.75\thumbheight}\selectfont
                \thechapter
            };
        \end{tikzpicture}%
    \fi%
}

% For whether we are creating a printed version or a digital version
\newif\ifdigital


%% Page style for empty pages.
\fancypagestyle{empty}{%
    \fancyhf{}
    \renewcommand*\headrulewidth{0pt}
    \renewcommand*\footrulewidth{0pt}
    \fancyhead{\cropmarks}
}

%% Page style for title pages.
\fancypagestyle{plain}{%
    \fancyhf{}
    \renewcommand*\headrulewidth{0pt}
    \renewcommand*\footrulewidth{0pt}
    \fancyhead{\cropmarks}
    \fancyfoot[C]{\thepage}
}

%% Fancy style for the main matter.
\fancypagestyle{mainmatter}{%
    \fancyhf{}
    %% Page numbers on the top left and top right.
    \fancyhead[LE]{\cropmarks\lthumb\thepage}
    \fancyhead[RO]{\cropmarks\rthumb\thepage}
    %% Chapter name on the left (even) page.
    \fancyhead[RE]{\nouppercase{\leftmark}}
    %% Section name on the right (odd) page.
    \fancyhead[LO]{\nouppercase{\rightmark}}
}

\pagestyle{mainmatter}
